% Options for packages loaded elsewhere
\PassOptionsToPackage{unicode}{hyperref}
\PassOptionsToPackage{hyphens}{url}
%
\documentclass[
]{article}
\usepackage{amsmath,amssymb}
\usepackage{lmodern}
\usepackage{iftex}
\ifPDFTeX
  \usepackage[T1]{fontenc}
  \usepackage[utf8]{inputenc}
  \usepackage{textcomp} % provide euro and other symbols
\else % if luatex or xetex
  \usepackage{unicode-math}
  \defaultfontfeatures{Scale=MatchLowercase}
  \defaultfontfeatures[\rmfamily]{Ligatures=TeX,Scale=1}
\fi
% Use upquote if available, for straight quotes in verbatim environments
\IfFileExists{upquote.sty}{\usepackage{upquote}}{}
\IfFileExists{microtype.sty}{% use microtype if available
  \usepackage[]{microtype}
  \UseMicrotypeSet[protrusion]{basicmath} % disable protrusion for tt fonts
}{}
\makeatletter
\@ifundefined{KOMAClassName}{% if non-KOMA class
  \IfFileExists{parskip.sty}{%
    \usepackage{parskip}
  }{% else
    \setlength{\parindent}{0pt}
    \setlength{\parskip}{6pt plus 2pt minus 1pt}}
}{% if KOMA class
  \KOMAoptions{parskip=half}}
\makeatother
\usepackage{xcolor}
\IfFileExists{xurl.sty}{\usepackage{xurl}}{} % add URL line breaks if available
\IfFileExists{bookmark.sty}{\usepackage{bookmark}}{\usepackage{hyperref}}
\hypersetup{
  pdftitle={worksheet01\_walker},
  pdfauthor={Reuben Walker},
  hidelinks,
  pdfcreator={LaTeX via pandoc}}
\urlstyle{same} % disable monospaced font for URLs
\usepackage[margin=1in]{geometry}
\usepackage{color}
\usepackage{fancyvrb}
\newcommand{\VerbBar}{|}
\newcommand{\VERB}{\Verb[commandchars=\\\{\}]}
\DefineVerbatimEnvironment{Highlighting}{Verbatim}{commandchars=\\\{\}}
% Add ',fontsize=\small' for more characters per line
\usepackage{framed}
\definecolor{shadecolor}{RGB}{248,248,248}
\newenvironment{Shaded}{\begin{snugshade}}{\end{snugshade}}
\newcommand{\AlertTok}[1]{\textcolor[rgb]{0.94,0.16,0.16}{#1}}
\newcommand{\AnnotationTok}[1]{\textcolor[rgb]{0.56,0.35,0.01}{\textbf{\textit{#1}}}}
\newcommand{\AttributeTok}[1]{\textcolor[rgb]{0.77,0.63,0.00}{#1}}
\newcommand{\BaseNTok}[1]{\textcolor[rgb]{0.00,0.00,0.81}{#1}}
\newcommand{\BuiltInTok}[1]{#1}
\newcommand{\CharTok}[1]{\textcolor[rgb]{0.31,0.60,0.02}{#1}}
\newcommand{\CommentTok}[1]{\textcolor[rgb]{0.56,0.35,0.01}{\textit{#1}}}
\newcommand{\CommentVarTok}[1]{\textcolor[rgb]{0.56,0.35,0.01}{\textbf{\textit{#1}}}}
\newcommand{\ConstantTok}[1]{\textcolor[rgb]{0.00,0.00,0.00}{#1}}
\newcommand{\ControlFlowTok}[1]{\textcolor[rgb]{0.13,0.29,0.53}{\textbf{#1}}}
\newcommand{\DataTypeTok}[1]{\textcolor[rgb]{0.13,0.29,0.53}{#1}}
\newcommand{\DecValTok}[1]{\textcolor[rgb]{0.00,0.00,0.81}{#1}}
\newcommand{\DocumentationTok}[1]{\textcolor[rgb]{0.56,0.35,0.01}{\textbf{\textit{#1}}}}
\newcommand{\ErrorTok}[1]{\textcolor[rgb]{0.64,0.00,0.00}{\textbf{#1}}}
\newcommand{\ExtensionTok}[1]{#1}
\newcommand{\FloatTok}[1]{\textcolor[rgb]{0.00,0.00,0.81}{#1}}
\newcommand{\FunctionTok}[1]{\textcolor[rgb]{0.00,0.00,0.00}{#1}}
\newcommand{\ImportTok}[1]{#1}
\newcommand{\InformationTok}[1]{\textcolor[rgb]{0.56,0.35,0.01}{\textbf{\textit{#1}}}}
\newcommand{\KeywordTok}[1]{\textcolor[rgb]{0.13,0.29,0.53}{\textbf{#1}}}
\newcommand{\NormalTok}[1]{#1}
\newcommand{\OperatorTok}[1]{\textcolor[rgb]{0.81,0.36,0.00}{\textbf{#1}}}
\newcommand{\OtherTok}[1]{\textcolor[rgb]{0.56,0.35,0.01}{#1}}
\newcommand{\PreprocessorTok}[1]{\textcolor[rgb]{0.56,0.35,0.01}{\textit{#1}}}
\newcommand{\RegionMarkerTok}[1]{#1}
\newcommand{\SpecialCharTok}[1]{\textcolor[rgb]{0.00,0.00,0.00}{#1}}
\newcommand{\SpecialStringTok}[1]{\textcolor[rgb]{0.31,0.60,0.02}{#1}}
\newcommand{\StringTok}[1]{\textcolor[rgb]{0.31,0.60,0.02}{#1}}
\newcommand{\VariableTok}[1]{\textcolor[rgb]{0.00,0.00,0.00}{#1}}
\newcommand{\VerbatimStringTok}[1]{\textcolor[rgb]{0.31,0.60,0.02}{#1}}
\newcommand{\WarningTok}[1]{\textcolor[rgb]{0.56,0.35,0.01}{\textbf{\textit{#1}}}}
\usepackage{graphicx}
\makeatletter
\def\maxwidth{\ifdim\Gin@nat@width>\linewidth\linewidth\else\Gin@nat@width\fi}
\def\maxheight{\ifdim\Gin@nat@height>\textheight\textheight\else\Gin@nat@height\fi}
\makeatother
% Scale images if necessary, so that they will not overflow the page
% margins by default, and it is still possible to overwrite the defaults
% using explicit options in \includegraphics[width, height, ...]{}
\setkeys{Gin}{width=\maxwidth,height=\maxheight,keepaspectratio}
% Set default figure placement to htbp
\makeatletter
\def\fps@figure{htbp}
\makeatother
\setlength{\emergencystretch}{3em} % prevent overfull lines
\providecommand{\tightlist}{%
  \setlength{\itemsep}{0pt}\setlength{\parskip}{0pt}}
\setcounter{secnumdepth}{-\maxdimen} % remove section numbering
\ifLuaTeX
  \usepackage{selnolig}  % disable illegal ligatures
\fi

\title{worksheet01\_walker}
\author{Reuben Walker}
\date{15 April 2024}

\begin{document}
\maketitle

\hypertarget{t-test}{%
\subsection{t-test}\label{t-test}}

\begin{Shaded}
\begin{Highlighting}[]
\CommentTok{\#Sample Vector}
\NormalTok{num\_samples }\OtherTok{\textless{}{-}} \DecValTok{10}
\CommentTok{\#Take a sample of random continues values between [{-}1,1] with .001 steps (number of steps 2000+1 for {-}1:1)}
\NormalTok{cont\_distribution }\OtherTok{\textless{}{-}} \FunctionTok{sample}\NormalTok{(}\FunctionTok{seq}\NormalTok{(}\SpecialCharTok{{-}}\DecValTok{1}\NormalTok{, }\DecValTok{1}\NormalTok{, }\AttributeTok{length.out =} \DecValTok{2001}\NormalTok{), }\AttributeTok{size =}\NormalTok{ num\_samples)}

\CommentTok{\#Take a sample of a student{-}t distribution with parameter nu = 1}
\CommentTok{\#To do this, we use the rt function (pseudo{-}random numbers from t{-}distribution) with nu (degrees of freedom) = 1}
\NormalTok{t\_distribution }\OtherTok{\textless{}{-}} \FunctionTok{rt}\NormalTok{(num\_samples, }\AttributeTok{df=}\DecValTok{1}\NormalTok{)}


\CommentTok{\#Take a sample of random discrete values either {-}1 or 1, replace=TRUE so we can keep sampling}
\NormalTok{disc\_distribution }\OtherTok{\textless{}{-}} \FunctionTok{sample}\NormalTok{(}\FunctionTok{c}\NormalTok{(}\SpecialCharTok{{-}}\DecValTok{1}\NormalTok{,}\DecValTok{1}\NormalTok{), }\AttributeTok{size=}\NormalTok{num\_samples, }\AttributeTok{replace=}\ConstantTok{TRUE}\NormalTok{)}

\CommentTok{\#Let\textquotesingle{}s just do it for n=5 and plot the histogram of the p values}
\NormalTok{df }\OtherTok{=} \FunctionTok{data.frame}\NormalTok{(}\AttributeTok{continuous=}\FunctionTok{double}\NormalTok{(),}
                \AttributeTok{t=}\FunctionTok{double}\NormalTok{(),}
                \AttributeTok{discrete=}\FunctionTok{double}\NormalTok{()}
\NormalTok{                )}

\ControlFlowTok{for}\NormalTok{ (j }\ControlFlowTok{in} \DecValTok{1}\SpecialCharTok{:}\DecValTok{10000}\NormalTok{) \{ }
\NormalTok{  num\_samples }\OtherTok{=} \DecValTok{5}
\NormalTok{  cont\_distribution\_1 }\OtherTok{\textless{}{-}} \FunctionTok{sample}\NormalTok{(}\FunctionTok{seq}\NormalTok{(}\SpecialCharTok{{-}}\DecValTok{1}\NormalTok{, }\DecValTok{1}\NormalTok{, }\AttributeTok{length.out =} \DecValTok{2001}\NormalTok{), }\AttributeTok{size =}\NormalTok{ num\_samples)}
\NormalTok{  cont\_distribution\_2 }\OtherTok{\textless{}{-}} \FunctionTok{sample}\NormalTok{(}\FunctionTok{seq}\NormalTok{(}\SpecialCharTok{{-}}\DecValTok{1}\NormalTok{, }\DecValTok{1}\NormalTok{, }\AttributeTok{length.out =} \DecValTok{2001}\NormalTok{), }\AttributeTok{size =}\NormalTok{ num\_samples)}
  \CommentTok{\#Pull the p{-}value from the t{-}test}
\NormalTok{  p\_cont }\OtherTok{\textless{}{-}} \FunctionTok{t.test}\NormalTok{(cont\_distribution\_1, cont\_distribution\_2)}\SpecialCharTok{$}\NormalTok{p.value}
  
\NormalTok{  t\_distribution\_1 }\OtherTok{\textless{}{-}} \FunctionTok{rt}\NormalTok{(num\_samples, }\AttributeTok{df=}\DecValTok{1}\NormalTok{)}
\NormalTok{  t\_distribution\_2 }\OtherTok{\textless{}{-}} \FunctionTok{rt}\NormalTok{(num\_samples, }\AttributeTok{df=}\DecValTok{1}\NormalTok{)}
\NormalTok{  p\_t }\OtherTok{\textless{}{-}} \FunctionTok{t.test}\NormalTok{(t\_distribution\_1, t\_distribution\_2)}\SpecialCharTok{$}\NormalTok{p.value}

\NormalTok{  disc\_distribution\_1 }\OtherTok{\textless{}{-}} \FunctionTok{sample}\NormalTok{(}\FunctionTok{c}\NormalTok{(}\SpecialCharTok{{-}}\DecValTok{1}\NormalTok{,}\DecValTok{1}\NormalTok{), }\AttributeTok{size=}\NormalTok{num\_samples, }\AttributeTok{replace=}\ConstantTok{TRUE}\NormalTok{)}
\NormalTok{  disc\_distribution\_2 }\OtherTok{\textless{}{-}} \FunctionTok{sample}\NormalTok{(}\FunctionTok{c}\NormalTok{(}\SpecialCharTok{{-}}\DecValTok{1}\NormalTok{,}\DecValTok{1}\NormalTok{), }\AttributeTok{size=}\NormalTok{num\_samples, }\AttributeTok{replace=}\ConstantTok{TRUE}\NormalTok{)}
  \CommentTok{\#The t.test returns an error when the two values are constant}
  \CommentTok{\#This is because there is a variance term in the denominator}
  \CommentTok{\#Check for null variance}
\NormalTok{  null\_variance }\OtherTok{\textless{}{-}}\NormalTok{ ((}\FunctionTok{var}\NormalTok{(disc\_distribution\_1) }\SpecialCharTok{==} \DecValTok{0}\NormalTok{) }\SpecialCharTok{\&}\NormalTok{ (}\FunctionTok{var}\NormalTok{(disc\_distribution\_2) }\SpecialCharTok{==} \DecValTok{0}\NormalTok{))}
\NormalTok{  p\_disc }\OtherTok{\textless{}{-}} \FunctionTok{ifelse}\NormalTok{(null\_variance, }\ConstantTok{NaN}\NormalTok{, }\FunctionTok{t.test}\NormalTok{(disc\_distribution\_1, disc\_distribution\_2)}\SpecialCharTok{$}\NormalTok{p.value)}
  
  \CommentTok{\#Append a row in the dataframe}
\NormalTok{  df }\OtherTok{\textless{}{-}} \FunctionTok{rbind}\NormalTok{(df, }\FunctionTok{data.frame}\NormalTok{(}\AttributeTok{continuous=}\NormalTok{p\_cont, }\AttributeTok{t=}\NormalTok{p\_t, }\AttributeTok{discrete=}\NormalTok{p\_disc))}
\NormalTok{\}}
\end{Highlighting}
\end{Shaded}

\hypertarget{histograms-for-continuous-student-t-and-discrete-distributions}{%
\subsection{Histograms for continuous, student-t, and discrete
distributions}\label{histograms-for-continuous-student-t-and-discrete-distributions}}

\#Bin width of 0.05, so the furthest left bar is p-value less than 0.05

\begin{Shaded}
\begin{Highlighting}[]
\FunctionTok{hist}\NormalTok{(df}\SpecialCharTok{$}\NormalTok{continuous)}\CommentTok{\#, breaks = seq(from=0, to=1, by=0.05))}
\end{Highlighting}
\end{Shaded}

\includegraphics{worksheet01_walker_files/figure-latex/unnamed-chunk-2-1.pdf}

\begin{Shaded}
\begin{Highlighting}[]
\FunctionTok{hist}\NormalTok{(df}\SpecialCharTok{$}\NormalTok{t)}\CommentTok{\#, breaks = seq(from=0, to=1, by=0.05))}
\end{Highlighting}
\end{Shaded}

\includegraphics{worksheet01_walker_files/figure-latex/unnamed-chunk-3-1.pdf}

\begin{Shaded}
\begin{Highlighting}[]
\FunctionTok{hist}\NormalTok{(df}\SpecialCharTok{$}\NormalTok{discrete)}\CommentTok{\#, breaks = seq(from=0, to=1, by=0.05))}
\end{Highlighting}
\end{Shaded}

\includegraphics{worksheet01_walker_files/figure-latex/unnamed-chunk-4-1.pdf}

\begin{Shaded}
\begin{Highlighting}[]
\CommentTok{\#How many is that for each?}
\NormalTok{cont\_result\_0 }\OtherTok{\textless{}{-}} \FunctionTok{sum}\NormalTok{(df}\SpecialCharTok{$}\NormalTok{continuous }\SpecialCharTok{\textless{}}\FloatTok{0.05}\NormalTok{)}\SpecialCharTok{/}\FunctionTok{length}\NormalTok{(df}\SpecialCharTok{$}\NormalTok{continuous)}
\NormalTok{cont\_result\_0}
\end{Highlighting}
\end{Shaded}

\begin{verbatim}
## [1] 0.0474
\end{verbatim}

\begin{Shaded}
\begin{Highlighting}[]
\CommentTok{\#0.04}

\NormalTok{t\_result\_0 }\OtherTok{\textless{}{-}} \FunctionTok{sum}\NormalTok{(df}\SpecialCharTok{$}\NormalTok{t }\SpecialCharTok{\textless{}}\FloatTok{0.05}\NormalTok{)}\SpecialCharTok{/}\FunctionTok{length}\NormalTok{(df}\SpecialCharTok{$}\NormalTok{t)}
\NormalTok{t\_result\_0}
\end{Highlighting}
\end{Shaded}

\begin{verbatim}
## [1] 0.0134
\end{verbatim}

\begin{Shaded}
\begin{Highlighting}[]
\CommentTok{\#0.01}

\CommentTok{\#Non{-}NaN subset}
\NormalTok{disc\_result\_0 }\OtherTok{\textless{}{-}} \FunctionTok{sum}\NormalTok{(}\FunctionTok{na.omit}\NormalTok{(df}\SpecialCharTok{$}\NormalTok{discrete) }\SpecialCharTok{\textless{}}\FloatTok{0.05}\NormalTok{)}\SpecialCharTok{/}\FunctionTok{length}\NormalTok{(}\FunctionTok{na.omit}\NormalTok{(df}\SpecialCharTok{$}\NormalTok{discrete))}
\NormalTok{disc\_result\_0}
\end{Highlighting}
\end{Shaded}

\begin{verbatim}
## [1] 0.01846648
\end{verbatim}

\begin{Shaded}
\begin{Highlighting}[]
\CommentTok{\#0.02}

\CommentTok{\#Now let\textquotesingle{}s repeat is for all sample sizes}
\NormalTok{sampleNumbers }\OtherTok{\textless{}{-}} \FunctionTok{c}\NormalTok{(}\DecValTok{5}\NormalTok{,}\DecValTok{10}\NormalTok{,}\DecValTok{20}\NormalTok{,}\DecValTok{50}\NormalTok{,}\DecValTok{100}\NormalTok{)}
\NormalTok{cont\_result }\OtherTok{\textless{}{-}} \FunctionTok{c}\NormalTok{()}
\NormalTok{t\_result }\OtherTok{\textless{}{-}} \FunctionTok{c}\NormalTok{()}
\NormalTok{disc\_result }\OtherTok{\textless{}{-}} \FunctionTok{c}\NormalTok{()}
\ControlFlowTok{for}\NormalTok{ (x }\ControlFlowTok{in}\NormalTok{ sampleNumbers) \{}
\NormalTok{  df }\OtherTok{=} \FunctionTok{data.frame}\NormalTok{(}\AttributeTok{continuous=}\FunctionTok{double}\NormalTok{(),}
                \AttributeTok{t=}\FunctionTok{double}\NormalTok{(),}
                \AttributeTok{discrete=}\FunctionTok{double}\NormalTok{()}
\NormalTok{                )}
  \ControlFlowTok{for}\NormalTok{ (j }\ControlFlowTok{in} \DecValTok{1}\SpecialCharTok{:}\DecValTok{10000}\NormalTok{) \{ }
\NormalTok{    num\_samples }\OtherTok{=}\NormalTok{ x}
\NormalTok{    cont\_distribution\_1 }\OtherTok{\textless{}{-}} \FunctionTok{sample}\NormalTok{(}\FunctionTok{seq}\NormalTok{(}\SpecialCharTok{{-}}\DecValTok{1}\NormalTok{, }\DecValTok{1}\NormalTok{, }\AttributeTok{length.out =} \DecValTok{20001}\NormalTok{), }\AttributeTok{size =}\NormalTok{ num\_samples)}
\NormalTok{    cont\_distribution\_2 }\OtherTok{\textless{}{-}} \FunctionTok{sample}\NormalTok{(}\FunctionTok{seq}\NormalTok{(}\SpecialCharTok{{-}}\DecValTok{1}\NormalTok{, }\DecValTok{1}\NormalTok{, }\AttributeTok{length.out =} \DecValTok{20001}\NormalTok{), }\AttributeTok{size =}\NormalTok{ num\_samples)}
    \CommentTok{\#Pull the p{-}value from the t{-}test}
\NormalTok{    p\_cont }\OtherTok{=} \FunctionTok{t.test}\NormalTok{(cont\_distribution\_1, cont\_distribution\_2)}\SpecialCharTok{$}\NormalTok{p.value}
    
\NormalTok{    t\_distribution\_1 }\OtherTok{\textless{}{-}} \FunctionTok{rt}\NormalTok{(num\_samples, }\AttributeTok{df=}\DecValTok{1}\NormalTok{)}
\NormalTok{    t\_distribution\_2 }\OtherTok{\textless{}{-}} \FunctionTok{rt}\NormalTok{(num\_samples, }\AttributeTok{df=}\DecValTok{1}\NormalTok{)}
\NormalTok{    p\_t }\OtherTok{=} \FunctionTok{t.test}\NormalTok{(t\_distribution\_1, t\_distribution\_2)}\SpecialCharTok{$}\NormalTok{p.value}
  
\NormalTok{    disc\_distribution\_1 }\OtherTok{\textless{}{-}} \FunctionTok{sample}\NormalTok{(}\FunctionTok{c}\NormalTok{(}\SpecialCharTok{{-}}\DecValTok{1}\NormalTok{,}\DecValTok{1}\NormalTok{), }\AttributeTok{size=}\NormalTok{num\_samples, }\AttributeTok{replace=}\ConstantTok{TRUE}\NormalTok{)}
\NormalTok{    disc\_distribution\_2 }\OtherTok{\textless{}{-}} \FunctionTok{sample}\NormalTok{(}\FunctionTok{c}\NormalTok{(}\SpecialCharTok{{-}}\DecValTok{1}\NormalTok{,}\DecValTok{1}\NormalTok{), }\AttributeTok{size=}\NormalTok{num\_samples, }\AttributeTok{replace=}\ConstantTok{TRUE}\NormalTok{)}
    \CommentTok{\#The lack of variation is much less likely with increasing sample size}
\NormalTok{    null\_variance }\OtherTok{\textless{}{-}}\NormalTok{ ((}\FunctionTok{var}\NormalTok{(disc\_distribution\_1) }\SpecialCharTok{==} \DecValTok{0}\NormalTok{) }\SpecialCharTok{\&}\NormalTok{ (}\FunctionTok{var}\NormalTok{(disc\_distribution\_2) }\SpecialCharTok{==} \DecValTok{0}\NormalTok{))}
\NormalTok{    p\_disc }\OtherTok{\textless{}{-}} \FunctionTok{ifelse}\NormalTok{(null\_variance, }\ConstantTok{NaN}\NormalTok{, }\FunctionTok{t.test}\NormalTok{(disc\_distribution\_1, disc\_distribution\_2)}\SpecialCharTok{$}\NormalTok{p.value)}
  
    
    \CommentTok{\#Append a row in the dataframe}
\NormalTok{    df }\OtherTok{\textless{}{-}} \FunctionTok{rbind}\NormalTok{(df, }\FunctionTok{data.frame}\NormalTok{(}\AttributeTok{continuous=}\NormalTok{p\_cont, }\AttributeTok{t=}\NormalTok{p\_t, }\AttributeTok{discrete=}\NormalTok{p\_disc))}
\NormalTok{  \}}
\NormalTok{  cont\_perc }\OtherTok{\textless{}{-}} \FunctionTok{sum}\NormalTok{(df}\SpecialCharTok{$}\NormalTok{continuous }\SpecialCharTok{\textless{}} \FloatTok{0.05}\NormalTok{)}\SpecialCharTok{/}\FunctionTok{length}\NormalTok{(df}\SpecialCharTok{$}\NormalTok{continuous)}
\NormalTok{  cont\_result }\OtherTok{\textless{}{-}} \FunctionTok{append}\NormalTok{(cont\_result, cont\_perc)}
  
\NormalTok{  t\_perc }\OtherTok{\textless{}{-}} \FunctionTok{sum}\NormalTok{(df}\SpecialCharTok{$}\NormalTok{t }\SpecialCharTok{\textless{}} \FloatTok{0.05}\NormalTok{)}\SpecialCharTok{/}\FunctionTok{length}\NormalTok{(df}\SpecialCharTok{$}\NormalTok{t)}
\NormalTok{  t\_result }\OtherTok{\textless{}{-}} \FunctionTok{append}\NormalTok{(t\_result, t\_perc)}
  
\NormalTok{  disc\_perc }\OtherTok{\textless{}{-}} \FunctionTok{sum}\NormalTok{(}\FunctionTok{na.omit}\NormalTok{(df}\SpecialCharTok{$}\NormalTok{discrete) }\SpecialCharTok{\textless{}} \FloatTok{0.05}\NormalTok{)}\SpecialCharTok{/}\FunctionTok{length}\NormalTok{(}\FunctionTok{na.omit}\NormalTok{(df}\SpecialCharTok{$}\NormalTok{discrete))}
\NormalTok{  disc\_result }\OtherTok{\textless{}{-}} \FunctionTok{append}\NormalTok{(disc\_result, disc\_perc)}
\NormalTok{\}}

\CommentTok{\#So more or less regardless of sample size, they\textquotesingle{}re staying the same?}
\FunctionTok{library}\NormalTok{(ggplot2)}
\FunctionTok{library}\NormalTok{(tidyverse)}
\end{Highlighting}
\end{Shaded}

\begin{verbatim}
## -- Attaching core tidyverse packages ------------------------ tidyverse 2.0.0 --
## v dplyr     1.1.4     v readr     2.1.5
## v forcats   1.0.0     v stringr   1.5.1
## v lubridate 1.9.3     v tibble    3.2.1
## v purrr     1.0.2     v tidyr     1.3.1
## -- Conflicts ------------------------------------------ tidyverse_conflicts() --
## x dplyr::filter() masks stats::filter()
## x dplyr::lag()    masks stats::lag()
## i Use the conflicted package (<http://conflicted.r-lib.org/>) to force all conflicts to become errors
\end{verbatim}

\begin{Shaded}
\begin{Highlighting}[]
\CommentTok{\#plot(sampleNumbers, cont\_result, col=\textquotesingle{}green\textquotesingle{}, ylim=c(0,0.1), main=\textquotesingle{}Probability of p\textless{}0.05 (continuous)\textquotesingle{})\#, type=\textquotesingle{}l\textquotesingle{}}
\CommentTok{\#plot(sampleNumbers, t\_result, type=\textquotesingle{}l\textquotesingle{}, col=\textquotesingle{}blue\textquotesingle{}, ylim=c(0,0.1))}
\CommentTok{\#plot(sampleNumbers, disc\_result, type=\textquotesingle{}l\textquotesingle{}, col=\textquotesingle{}red\textquotesingle{}, ylim=c(0,1))}

\CommentTok{\#And all together?}
\NormalTok{plot\_df }\OtherTok{\textless{}{-}} \FunctionTok{data.frame}\NormalTok{(}\AttributeTok{samplesize=}\NormalTok{sampleNumbers, }
                      \AttributeTok{continuous=}\NormalTok{cont\_result,}
                      \AttributeTok{t=}\NormalTok{t\_result,}
                      \AttributeTok{discrete=}\NormalTok{disc\_result)}
                      
\FunctionTok{library}\NormalTok{(tidyverse)}
\NormalTok{plot\_df2 }\OtherTok{\textless{}{-}}\NormalTok{ plot\_df }\SpecialCharTok{\%\textgreater{}\%}
  \FunctionTok{select}\NormalTok{(samplesize, continuous, t, discrete) }\SpecialCharTok{\%\textgreater{}\%}
  \FunctionTok{gather}\NormalTok{(}\AttributeTok{key =} \StringTok{"variable"}\NormalTok{, }\AttributeTok{value =} \StringTok{"value"}\NormalTok{, }\SpecialCharTok{{-}}\NormalTok{samplesize)}
\FunctionTok{head}\NormalTok{(plot\_df2)}
\end{Highlighting}
\end{Shaded}

\begin{verbatim}
##   samplesize   variable  value
## 1          5 continuous 0.0500
## 2         10 continuous 0.0493
## 3         20 continuous 0.0484
## 4         50 continuous 0.0529
## 5        100 continuous 0.0520
## 6          5          t 0.0156
\end{verbatim}

\begin{Shaded}
\begin{Highlighting}[]
\NormalTok{plot\_f }\OtherTok{\textless{}{-}}\FunctionTok{ggplot}\NormalTok{(plot\_df2, }\FunctionTok{aes}\NormalTok{(}\AttributeTok{x =}\NormalTok{ samplesize, }\AttributeTok{y =}\NormalTok{ value)) }\SpecialCharTok{+} 
  \FunctionTok{geom\_line}\NormalTok{(}\FunctionTok{aes}\NormalTok{(}\AttributeTok{color =}\NormalTok{ variable, }\AttributeTok{linetype =}\NormalTok{ variable)) }\SpecialCharTok{+} 
  \FunctionTok{geom\_point}\NormalTok{(}\FunctionTok{aes}\NormalTok{(}\AttributeTok{color =}\NormalTok{ variable)) }\SpecialCharTok{+}
  \FunctionTok{scale\_color\_manual}\NormalTok{(}\AttributeTok{values =} \FunctionTok{c}\NormalTok{(}\StringTok{"darkred"}\NormalTok{, }\StringTok{"steelblue"}\NormalTok{, }\StringTok{"purple"}\NormalTok{))}
\NormalTok{plot\_f }\SpecialCharTok{+} \FunctionTok{ggtitle}\NormalTok{(}\StringTok{"Percentage of (p\textless{}0.05) Over Increasing Sample Sizes for Differing Distributions"}\NormalTok{) }\SpecialCharTok{+} \FunctionTok{xlab}\NormalTok{(}\StringTok{"Sample Size"}\NormalTok{) }\SpecialCharTok{+} \FunctionTok{ylab}\NormalTok{(}\StringTok{"Percentage p values \textless{} 0.05"}\NormalTok{)}
\end{Highlighting}
\end{Shaded}

\includegraphics{worksheet01_walker_files/figure-latex/unnamed-chunk-6-1.pdf}
Conclusion: We choose p values less than 0.05 because in a uniform
continuous distribution we would expect to see ``significant'' results
5\% of the time. We would choose p values less than 0.05 if we suspected
our distribution to be a t-distribution of some sort.

\end{document}
